%%%%%%%%%%%%%%%%%%%%%%%%%%%%%%
%Documentclass Einstellungen %
%%%%%%%%%%%%%%%%%%%%%%%%%%%%%%

\documentclass[
    a4paper, %Papiergröße
    12pt, %Schriftgröße
    bibliography=totocnumbered, %Literaturverzereichnis als Eintrag ins Inhaltsverzeichnis
    twoside, %Zweiseitiger Druck
    %oneside, %Einseitiger Druck	
    BCOR=1cm, %Platz zum Lochen
]{scrartcl}


%%%%%%%%%%%%%%%%%%%%%%%%%%%%%%%%%%%%%%
% Geladene Dateien, für mehr ordnung %
%%%%%%%%%%%%%%%%%%%%%%%%%%%%%%%%%%%%%%
\usepackage{properties/packages, properties/commands, properties/preferences, properties/details}
\graphicspath{ {ressources/} }
% Zu beachten ist:
% >> packages sind alle im Protokoll benutzen Packete
% >> preferences sind alle Einstellungen des Protokolls wie pagelayout, packeteinstellungen, usw
% >> commands sind eigen erstellte commands, um das lange tippen zu sparen
% >> details hier befinden sich die Details, wie Versuchsdatum, Assistent, Teilnehmer, Versuchsnummer, usw. muss unbedingt ausgefüllt werden!





%%%%%%%%%%%%
% Dokument %
%%%%%%%%%%%%

\begin{document}
	
	
	%%%%%%%%%%%%%%
	% Titelseite %
	%%%%%%%%%%%%%%
	\thispagestyle{empty} % Auf Titelseite kein pagestyle, sodass man diese selbst gestalten kann
	
	\begin{titlepage}
		% Die ganzen Makros wie \VERSUCHSNAME, \VerfasserEINS, usw sind in der Datei details.sty auszufüllen!
		\begin{center}
			\Huge{\textbf{\VERSUCHSNR\ - \VERSUCHSNAME}}\\% \Huge \huge \Large \normalsize \Small usw. bestimmt die Schriftgröße.
			\vspace{10mm}% Abstand
			\Large{Protokoll zum Versuch des Physikalischen Praktikums I von \\ \textbf{\VerfasserEINS\;\& \VerfasserZWEI}}\\
			\vspace{10mm} 
			\Large{Universität Stuttgart}\\
		\end{center}
		\vspace{1cm}
		\begin{center}
			\begin{tabular}{ll}
				\large{Verfasser:}		& \large{\VerfasserEINS\;(\StudiengangEINS),} \\ 
				& \large{\MatNoEINS} \\
				\vspace{0cm}\\
				& \large{\VerfasserZWEI\;(\StudiengangZWEI),} \\
				& \large{\MatNoZWEI} \\
				\vspace{0cm}\\
				\large{Gruppennummer:}	& \large{\GRUPPENNR} \\
				\vspace{0cm}\\
				\large{Versuchsdatum:}	& \large{\VERSUCHSDATUM} \\
				\vspace{0cm}\\
				\large{Betreuer:}		& \large{\BETREUER}
			\end{tabular}
		\end{center}
		\vspace{15mm}
		
		\begin{center}
			Stuttgart, den \PROTOKOLLDATUM
		\end{center}
		
	\end{titlepage}
	
	
	%%%%%%%%%%%%%%%%%%%%%%%
	% Inhaltsverzeichniss %
	%%%%%%%%%%%%%%%%%%%%%%%
	
	\thispagestyle{empty}
	
	\tableofcontents 
	
	\clearpage %Neue Seite, davor werden alle noch ausstehenden Grafiken/Tabellen platziert.
	
	% Ab hier wollen wir nochmals neu die Seiten nummerieren, es fängt wieder mit Seite 1. an
	\renewcommand{\thepage}{\arabic{page}}
	\setcounter{page}{1}
	



    
	%%%%%%%%%%%%%%%%%%%%
	% Beginn Protokoll %
	%%%%%%%%%%%%%%%%%%%%
	
	
	%%%%%%%%%%%%%%%%%
	% Versuchsziele %
	%%%%%%%%%%%%%%%%%
	
	% Die erste eckige Klammer ist optional, die darin angegebene Bezeichnung steht im Inhaltsverzeichnis anstelle des hinteren (längeren) Namens.
	\section[Versuchsziel]{Versuchsziel und Versuchsmethode}
        \SI{3}{\meter\per\second}
	
	%%%%%%%%%%%%%%
	% Grundlagen %
	%%%%%%%%%%%%%%
	
	\section[Grundlagen]{Grundlagen}
	
	
	%%%%%%%%%%%%%%%%%%
	% Messprinzipien %
	%%%%%%%%%%%%%%%%%%
	
	\section[Messprinzip]{Messprinzip mit Skizze und Versuchsablauf}
	
    	\subsection{Aufbau und Geräte}
            % Hier wird der Grobe aufbau aufgelistet, an besten mit Skizze
    	
    	\subsection{Versuch} 
	
	
	%%%%%%%%%%%%%
	% Messwerte %
	%%%%%%%%%%%%%
	
	\section[Messwerte]{Messwerte}
        % Messwerte werden in Tabellen eingetragen, Caption ist über der Tabelle und sie soll auch ein wenig beschrieben werden.
	
	
	%%%%%%%%%%%%%%
	% Auswertung %
	%%%%%%%%%%%%%%
	
	\section{Auswertung}
	
	
	%%%%%%%%%%%%%%%%%%
	% Fehlerrechnung %
	%%%%%%%%%%%%%%%%%%
	
	\section[Fehlerbetrachtung]{Fehlerbetrachtung}

	
	%%%%%%%%%%%%%%%%%%%
	% Zusammenfassung %
	%%%%%%%%%%%%%%%%%%%
	
	\section[Zusammenfassung]{Zusammenfassung}
	
	
	\clearpage
	%%%%%%%%%%%%%%%%%%%%%%%%%%
	%%%Literaturverzeichnis%%%
	%%%%%%%%%%%%%%%%%%%%%%%%%%
	
	\section{Literatur}
	\begin{thebibliography}{999}
		\bibitem{Quelle} Versuchsanleitung zu (Abgerufen am 1.04.2050) 
	\end{thebibliography}
	
	
	\clearpage
	%%%%%%%%%%
	% Anhang %
	%%%%%%%%%%
	
    \section{Anhang}
        % Hier kommt das Messwertblatt rein
        %\includepdf[pages=-]{Dateiname.pdf}
	
\end{document}